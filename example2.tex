\documentclass{article}

\usepackage{clrscode}
\usepackage{graphicx}
\usepackage{psfig}

\title{Recursion Trees in Latex}
\author{Martin~J.~Strauss}

\begin{document}

\maketitle

\section{Overview}

There are several choices for giving recurrence trees in latex:
\begin{itemize}
\item latex \verb|tabular| or \verb|array| environment
\item including a postscript or pdf file generated elsewhere
\item latex \verb|picture| environment
\item a package for trees (similar in concept to a package for
  pseudocode)
\end{itemize}


\section{Table Example}

To describe a recursion tree in Latex, one can use a table.  To
typeset a table in text mode, use the \verb|tabular| environment (and
all math has to go in math mode, between dollar signs).  To typeset a
table in math mode, use the \verb|array| environment (and all text needs to
go in an \verb|\mbox|).

\[
\begin{array}{|c|c|c|c|}  	% four columns; position text in CENTER of column 
\hline				% use `r' for right and `l' for left
\mbox{Depth} & \mbox{Number of nodes} & \mbox{Problem size (each node)} & \mbox{ total problem size}\\
\hline\hline
% depth   & number        & size & total size
  0       & 1             &  n   & n\\
\hline
  1       & 3             & n/2  & 3n/2\\
\hline
  2       & 9             & n/4  & 9n/4\\
\hline
 \vdots & & & \\
\hline
\log_2(n) & 3^{\log_2(n)} & 1    & 3^{\log_2(n)}\\
\hline
\end{array}
\]

\section{\LaTeX Picture Enviornment}
 
\begin{center}
\begin{picture}(200,100)(0,100) % 200 horiz by 100 vertical picture.
                                % lower left corner has coordinates (0,100)

\put(100,200){\makebox(0,0)[c]{$n$}}
\put( 90,190){\vector(-2,-1){20}}
\put(110,190){\vector(+2,-1){20}}

\put( 60,160){\makebox(0,0)[c]{$n/3$}}
\put(140,160){\makebox(0,0)[c]{$n/3$}}

\put( 55,150){\vector(-1,-2){5}}
\put( 65,150){\vector(+1,-2){5}}

\put(135,150){\vector(-1,-2){5}}
\put(145,150){\vector(+1,-2){5}}

\put( 45, 130){\makebox(0,0)[c]{$n/9$}}
\put( 75, 130){\makebox(0,0)[c]{$n/9$}}
\put(125, 130){\makebox(0,0)[c]{$n/9$}}
\put(155, 130){\makebox(0,0)[c]{$n/9$}}

\put( 45, 120){\makebox(0,0)[c]{$\vdots$}}
\put( 75, 120){\makebox(0,0)[c]{$\vdots$}}
\put(125, 120){\makebox(0,0)[c]{$\vdots$}}
\put(155, 120){\makebox(0,0)[c]{$\vdots$}}

\put( 45, 100){\makebox(0,0)[c]{$1$}}
\put( 75, 100){\makebox(0,0)[c]{$1$}}
\put(125, 100){\makebox(0,0)[c]{$1$}}
\put(155, 100){\makebox(0,0)[c]{$1$}}

\put(0,145){\makebox(0,0)[c]{$\log(n)$}}
\put(0,155){\vector(0,1){35}}
\put(0,135){\vector(0,-1){35}}

\end{picture}
\end{center}

\section{Package}

You might try a package like:

\verb|http://www.essex.ac.uk/linguistics/clmt/latex4ling/trees/|

I'm not familiar with it.

\section{Including a Picture}

See \verb|http://www.artofproblemsolving.com/LaTeX/AoPS_L_PictHow.php|

These instruction vary from installation to installation.  While I
could not get that to work, the following did.  On linux, I did the
following by creating a file using xfig, exporting to \verb|tree.eps|,
using the command \verb|latex example2| to get \verb|example2.dvi|,
then converting to \verb|example2.pdf| by \verb|dvipdfm example2|.
The included tree picture is in Figure~\ref{fig:sine}.

(latex is very good for typesetting mathematics and related technical
material, but not very good for simple images.)

\begin{figure}
\caption{Included Figure.}\label{fig:sine}
\vspace*{12pt}
\centerline{\psfig{figure=tree.eps,width=3.0in}}
\end{figure}

\end{document}
